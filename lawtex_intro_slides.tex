\documentclass{beamer}
\usepackage{listings}
\usetheme{Montpellier}
\usecolortheme{beaver}

\renewcommand\mathfamilydefault{\rmdefault} % Beamer uses san serif font for math; unfortunately unavailable as METAFONT. Do this to use default Computer Modern.

\title{The best \LaTeX\ editor}
\author{Lightech \and LawTeX}
\date{Till The End of Time} % Set date explicitly; without this, we will run into undefined control sequence \languagename at \maketitle. This might be due to defect in our set-up.

\begin{document}

\maketitle

\begin{frame}{What is LawTeX?}
\begin{itemize}
\item End to end \LaTeX\ solution
\begin{itemize}
\item A highly optimized source code editor written in C++ exploiting the graphic powerhouse DirectX
\item The original i.e. D. E. Knuth's \TeX\ compiler to produce DVI output
\item A (DirectX based) DVI viewer, optimized for live rendering
\end{itemize}
\item Optimized Universal Windows Platform (UWP) app
\begin{itemize}
\item Designed for Windows 10 devices
\item Available for PC, tablets, phones, HoloLens, XBox, etc.
\item Continuum--ready
\end{itemize}
\end{itemize}
\end{frame}

\begin{frame}{Goal of the app}
The best environment to \emph{write} documents and \emph{prepare} presentations
\begin{itemize}
\item For publishing print--ready documents, consider pdfTeX or online tools like ShareLatex.com
\item (We love to add support for ShareLatex so that you can backup your documents there but the site does not provide an API at the moment.)
\item (And no, we do not plan to include DVI to PDF converter.)
\end{itemize}
\end{frame}

\begin{frame}{Features}
\begin{itemize}
\item Automatically detect missing packages
\item Extremely small footprint
\begin{itemize}
\item Takes a breeze to install and ready to compose your document
\item The app itself is about 1.2MB download
\item Typical packages (for math articles) take up about 8MB
\item In our installation, it takes 24MB in total \emph{with beamer and powerdot}
\end{itemize}
\item Option to automatically compile and re--render the DVI when you pause typing
\begin{itemize}
\item No need to waste your time pressing F9 (WinEdt), switching back and forth from Adobe Reader
\end{itemize}
\end{itemize}
\end{frame}

\begin{frame}{Features}
Being modern app
\begin{itemize}
\item Support additional input methods
\begin{itemize}
\item Example: touch screen
\item Example: input text using the word flow keyboard (on phones only), voice, handwriting input panel (PC only)
\end{itemize}
\item Optimizes for battery usage unlike traditional Win32 PC app
\begin{itemize}
\item For example, graphic memory is freed when suspending app
\end{itemize}
\item Low memory (RAM) usage
\begin{itemize}
\item Typically consumes about 30-40MB while TeX Maker consumes 60-100MB (not to count the PDF reader)
\end{itemize}
\item Faster compilation
\begin{itemize}
\item We do not need to load the engines over and over again
\end{itemize}
\item Easy to uninstall; without polluting your system
\end{itemize}
\end{frame}

\begin{frame}{Limitation}
\begin{itemize}
\item Cannot support some packages such as \texttt{moderncv}, \texttt{context} due to their dependency on some modern TeX engines such as pdfTeX or LuaTeX
\item Cannot support bitmap graphics due to \TeX\ limitation; cannot handle EPS graphics since we have no PostScript processing capability
\item Cannot input Tab in the editor
\begin{itemize}
\item \emph{Reason}: The UWP model does not allow apps to handle Tab key which is used to switch between UI elements such as text fields or buttons
\end{itemize}
\item Cannot support \texttt{\textbackslash input}, \texttt{\textbackslash include}
\begin{itemize}
\item \emph{Reason}: The UWP model does not provide access to the file system freely. For an app to access a file, user needs to select it from the UI.
\end{itemize}
\end{itemize}
\end{frame}

\begin{frame}{Usage Tip: Conditional Compilation}
\textbf{Problem}
\begin{itemize}
\item LawTeX uses original \TeX\ which generates DVI while you probably want to use pdfTeX to get the published version
\item Some \TeX\ features are only available in pdfTeX while others are also present in the original
\item Wish to use lighter weight source code when editing (for faster recompilation) and heavy weight (but much more beautiful result) when publishing
\end{itemize}
\textbf{Solution}
\begin{itemize}
\item Use control sequences such as \texttt{\textbackslash ifPDFTeX} in package \texttt{iftex} to switch between engines in a single source code
\end{itemize}
\end{frame}

\begin{frame}{Usage Tip: Missing Fonts}
\textbf{Problem}
\begin{itemize}
\item Our DVI rendering engine does not support Type 1 fonts that many fonts are distributed
\item Only fonts based on Knuth's METAFONT can be rendered
\end{itemize}
\textbf{Solution}
\begin{itemize}
\item Search for alternatives on the web
\item Use conditional compilation to use METAFONT alternative on LawTeX
\end{itemize}
\textbf{Example}
\begin{itemize}
\item \texttt{\textbackslash mathfrak} in \texttt{amsmath} package resolves to Euler Frakture font. The AMS decided to remove the METAFONT version and distributed only the Type 1 version (for use in pdfTeX). As an alternative, use \texttt{\textbackslash textfrak} in \texttt{yfonts} package.
\end{itemize}
\end{frame}

\begin{frame}[fragile]{Usage Tip: Missing Fonts}
\begin{itemize}
\item Using conditional compilation, do something like
\lstset{language=TeX, morekeywords={ifPDFTeX,RequirePackage,usepackage,newcommand}}
\begin{lstlisting}
\RequirePackage{iftex}
\ifPDFTeX
    % Do nothing since \mathfrak is
    % available in pdfTeX
\else
    \usepackage{yfonts}
    % This will allow us to use a single
    % \mathfrak command in either case
    % for Gothic characters
    \newcommand{\mathfrak}{\textfrak}
\fi
\end{lstlisting}
\end{itemize}
\end{frame}

\begin{frame}{Thanks for choosing LawTeX and $\cdots$}
\begin{center}
\Huge Happy \TeX ing
\end{center}
\end{frame}

\end{document}